
%\addcontentsline{toc}{chapter}{Abstract}

\begin{abstract}
\par
Στα πλαίσια της εργασίας αυτής διερευνήθηκε η συμπεριφορά και η απόδοση του αλγορίθμου <<Τοπική Γραμμική Ενσωμάτωση>>\cite{lle} στο πεδίο της Αναγνώρισης Προτύπων. Ο αλγόριθμος ανήκει στην ευρύτερη κατηγορία <<Αλγόριθμοι Μείωσης των Διαστάσεων>> με τους οποίους μπορούμε να επιτύχουμε μείωση των παραμέτρων οι οποίες προσδιορίζουν κάποιο συγκεκριμένο πρόβλημα. Οι βασικοί μας στόχοι μέσω αυτής της διαδικασίας είναι αρχικά η συμπίεση της πληροφορίας, δηλαδή η δυνατότητα να εκφράσουμε την πληροφορία των αρχικών μας δεδομένων με ένα υποσύνολο της, με τις ελάχιστες δυνατές απώλειες. Επίσης μπορούμε να επιτύχουμε τεράστια μείωση της υπολογιστικής πολυπλοκότητας αλλά και της διαθέσιμης μνήμης που απαιτούνται για την προσπέλαση,αποθήκευση και μετέπειτα επεξεργασία των δεδομένων. Τέλος υπάρχουν περιπτώσεις στις οποίες θέλουμε να απομακρύνουμε απο τα δεδομένα μας, στοιχεία τα οποία αποτελούν θόρυβο και επιδρούν αρνητικά στην εξαγωγή ορθού συμπεράσματος ταξινόμησης. 
\par
Πιο συγκεκριμένα έγινε εφαρμογή της διαδικασίας μείωσης των διαστάσεων μέσω του αλγορίθμου <<Τοπική Γραμμική Ενσωμάτωση>>\cite{lle} σε τρία σετ δεδομένων. Τα δύο πρώτα περιέχουν εικόνες με ψηφία-αριθμούς και είναι τα \textlatin{MNIST}\cite{16} και \textlatin{Google Streen View House Numbers}\cite{12}. Το τρίτο είναι το \textlatin{Arcene}\cite{15} και περιέχει δεδομένα απο τον χώρο της Ιατρικής και συγκεκριμένα πρόκειται για δεδομένα απο ασθενείς με σκοπό την πρόβλεψη εμφάνισης κάποιας μορφής καρκίνου. Στα πρώτα δύο ο στόχος μας είναι να γίνει σωστή αναγνώριση κάθε ψηφίου. 
\par
Ο τελικός σκοπός είναι να προσδιορίσουμε με ακρίβεια κατά πόσο μπορούμε να επιτύχουμε συμπίεση της πληροφορίας και τι επιδράσεις θα έχει αυτό στην διαδικασία της ταξινόμησης των δεδομένων σε κλάσεις. Μέσα απο τα πειράματα λοιπόν έγινε προσπάθεια να διερευνηθεί τόσο η αποτελεσματικότητα του αλγορίθμου στα διαφορετικά σετ δεδομέναν αλλά και την επίδραση που έχουν οι παράμετροί του στην επίλυση κάθε προβλήματος χωριστά. Ως μετρική αξιολόγησης της αποτελεσματικότητας του αλγορίθμου χρησιμοποιήθηκε η σύγκριση μεταξύ του σφάλματος ταξινόμησης πρίν και μετά την διαδικασία μείωσης των διαστάσεων.
\par
Πολύ σημαντικό εύρημα της εν λόγω δουλειάς πέραν των πολύ ικανοποιητικών αποτελεσμάτων μετά την μείωση των διαστάσεων είναι η παρουσίαση δύο νέων μεθόδων, οι οποίες αποτελούν παραλλαγές του αλγορίθμου <<Τοπική Γραμμική Ενσωμάτωση>>\cite{lle}. Με την πρώτη μέθοδο γίνεται εφικτή η χρήση του αλγορίθμου σε προβλήματα ταξινόμησης όπου τα αποτελέσματα θα πρέπει να δίνονται σε <<πραγματικό χρόνο>>, μειώνοντας παράλληλα και την πολυπλοκότητα εκτέλεσης του αλγορίθμου. Με την δεύτερη μέθοδο μειώνεται δραματικά το πολύ μεγάλο υπολογιστικό κόστος που απαιτεί ο αλγόριθμος κατά την εκτέλεσή του. Τέλος, πολύ σημαντικό στοιχείο αποτελεί το γεγονός ότι οι δύο αυτές μέθοδοι μπορούν να συνδιαστούν μεταξύ τους έχοντας έτσι πολλαπλή μείωση της πολυπλοκότητας και άμεση εξαγωγή των αποτελεσμάτων ταξινόμησης.  


\end{abstract}
