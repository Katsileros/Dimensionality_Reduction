
%\addcontentsline{toc}{chapter}{Abstract}

\begin{abstract}
\par Η τεχνητή νοημοσύνη μέσω της μηχανικής μάθησης είναι αναμφισβήτητα ένας επιστημονικός κλάδος ο οποίος επικεντρώνει το ενδιαφέρον ολοένα και περισσότερων μηχανικών-ερευνητών. Το γεγονός αυτό οφείλεται στην επιτυχία τέτοιου είδους εφαρμογών σε διάφορους κλάδους της καθημερινότητάς μας όπως αυτός της ρομποτικής, της υγείας, της εξόρυξης γνώσης κλπ. Επίσης οι σημερινοί υπολογιστές λόγω της ραγδαίας ανάπτυξης της τεχνολογίας παρέχουν τους απαραίτητους πόρους ώστε να μπορέσουν να αναπτυχθούν και να διερευνηθούν τέτοιου είδους προβλήματα. Παρ' όλα αυτά, όσους πόρους και αν διαθέσουμε δεν μπορούμε σε καμιά περίπτωση να δημιουργήσουμε κάτι αντίστοιχο με τον ανθρώπινο εγκέφαλο. 
\par Γνωρίζουμε οτι ο ανθρώπινος εγκέφαλος είναι ένα τρομερά περίπλοκο σύστημα εκατομμυρίων νευρώνων συνδεδεμένων μεταξύ τους οι οποίοι είναι σε θέση να εκτελούνε σε κλάσματα του δευτερολέπτου έναν τεράστιο αριθμό λογικών πράξεων. Το μοντέλο αυτό είναι αδύνατον να προσομοιωθεί με οποιοδήποτε υπολογιστικό σύστημα διαθέτει ο άνθρωπος σήμερα. Στην προσπάθεια των Μηχανικών να μοντελοποιήσουν τις λειτουργίες του λαμβάνοντας φυσικά υπόψιν ευρήματα και αποτελέσματα των επιστημόνων της Ιατρικής σημαντικές λύσεις και βελτιστοποιήσεις έρχονται να δώσουν αλγόριθμοι οι οποίοι έχουν ως στόχο να μειώσουν τις παραμέτρους τις οποίες πρέπει να εκτιμήσει κάποιο υπολογιστικό σύστημα ώστε τελικά να μπορέσει να εξάγει συμπεράσματα, αντίστοιχα ενός ανθρώπου. 
\par Χαρακτηριστικά παραδείγματα τέτοιων εφαρμογών με τα οποία καταπιάνεται και η εργασία αυτή είναι η μείωση των παραμέτρων σε μοτίβα εικόνων ή άλλων δεδομένων με στόχο την εξαγωγή συμπεράσματος για την ταξινόμηση των δεδομένων σε κλάσεις. Συγκεκρίμένα γίνεται εφαρμογή του αλγορίθμου \textlatin{Locally Linear Embedding} σε τρία σετ δεδομένων. Τα δύο περιέχουν εικόνες με ψηφία αριθμών και στόχος είναι να γίνει ταξινόμηση των ψηφίων αυτών, και το τελευταίο σετ περιέχει ιατρικά δεδομένα ασθενών με σκοπό την πρόβλεψη εμφάνισης ή όχι κάποιας μορφής καρκίνου. Αφού γίνει μείωση των διαστάσεων οι οποίες και λαμβάνονται τελικά υπόψιν για την εξαγωγή του τελικού συμπεράσματος, εφαρμόζεται ο ταξινομητής κοντινότερων γειτόνων \textlatin{(K-Nearest Neighbors (KNN))} ο οποίος εξάγει και το τελικό συμπέρασμα για την ταξινόμηση των δεδομένων στις κατάλληλες κλάσεις.

\end{abstract}
