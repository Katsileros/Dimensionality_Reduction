
\chapter{Ο αλγόριθμος \textlatin{Locally Linear Embeddings - LLE}}
\numberwithin{equation}{section}
\par
Ο αλγόριθμος \textlatin{LLE} ανήκει στην κατηγορία αλγορίθμων μη γραμμικής μείωσης διαστάσεων με την χρήση γράφων και αποτελεί μια απο τις αποτελεσματικότερες αλλά και γρηγορότερες τεχνικές αυτού του είδους. Όπως αναφέραμε και στο προηγούμενο κεφάλαιο, βασική υπόθεση της μεθόδου είναι ότι τα δεδομένα μας βρίσκονται σε μια αρκετά λεία πολλαπλότητα, διάστασης $m$, και η οποία είναι ενσωματωμένη στον υποχώρο του $ \Re^{N} $, με $m<N$. Η υπόθεση για το λείο της πολλαπλότητας μας επιτρέπει να υποθέσουμε επιπλέον ότι, με δεδομένη την ύπαρξη αρκετών δεδομένων και ότι η πολλαπλότητα είναι $\ll$καλά$\gg$ δειγματοληπτημένη, τα κοντινά σημεία βρίσκονται πάνω (ή κοντά) σε ένα $\ll$τοπικό$\gg$ γραμμικό τμήμα της πολλαπλότητας.

\section{Ο αλγόριθμος ως τεχνική μη γραμμικής μείωσης διαστάσεων}
\par
Δεδομένης της αποτελεσματικότητας του αλγορίθμου να ανακαλύπτει τον χώρο μειωμένης διάστασης στον οποίο βρίσκεται ενσωματωμένη η πληροφορία ενός προβλήματος, ο αλγόριθμος έχει χρησιμοποιηθεί με επιτυχία σε αρκετές πρακτικές εφαρμογές.\cite{einstein} \cite{Mayer}


\section{Μαθηματική ανάλυση και υλοποίηση του αλγορίθμου \textlatin{Locally Linear Embeddings} σε κώδικα \textlatin{MATLAB}}
\subsection{Βήμα-1: Εύρεση του πίνακα γειτνίασης}

\subsection{Βήμα-2: Κατασκευή του \textlatin{Laplacian}}

\subsection{Βήμα-3: Επίλυση του προβλήματος εύρεσης ιδιτιμών και ιδιοδιανυσμάτων}

\subsection{Βήμα-4: Επιλογή των τελικών διαστάσεων}

