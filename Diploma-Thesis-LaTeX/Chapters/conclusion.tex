\chapter{Συμπεράσματα}
\par
Λαμβάνοντας υπόψιν τα αποτελέσματα των πειραμάτων τα οποία πραγματοποιήθηκαν στα πλαίσια της εν λόγω διατριβής έχουμε πλέον τα απαραίτητα στοιχεία απο τα οποία γίνεται φανερό ότι η μείωση των διαστάσεων αποτελεί έναν πολύ σημαντικό παράγοντα τόσο στην επίτευξη καλύτερων αποτελεσμάτων όσο και στην δρματική μείωση της πολυπλοκότητας των υπολογισμών κατά την διαδικασία της εξαγωγής του αποτελέσματος ταξινόμησης. Επίσης, απο την διαδικασία αυτή προκύπτουν σημαντικά ευρήματα και για τον αλγόριθμο μη γραμμικής μείωσης διαστάσεων \textlatin{LLE}. Πιο συγκεκριμένα στο πρώτο πείραμα με το σετ δεδομένων \textlatin{MNIST} χρησιμοποιήθηκε ο αλγόριθμος \textlatin{LLE} ουσιαστικά για την εξαγωγή συγκεκριμένου αριθμού χαρακτηριστικών πάνω στα \textlatin{pixel} κάθε εικόνας. Το τελικό αποτέλεσμα είναι ένα διάνυσμα μήκους \textlatin{d} (οι τελικές διαστάσεις του αλγορίθμου) για κάθε εικόνα. Όπως φάνηκε απο τα πειράματα το διάνυσμα αυτό περιέχει το σύνολο της πληροφορίας την οποία χρειαζόμαστε για την ταξινόμηση των δεδομένων στην κατάλληλη κλάση. Σε συγκεκριμένες περιπτώσεις μάλιστα φάνηκε ότι η διαδικασία της μείωσης των διαστάσεων μπορεί να βελτιώσει το ποσοστό σφάλματος στο τελικό βήμα της ταξινόμησης. 
\par
Στο δεύτερο πείραμα εφαρμόστηκε ο αλγόριθμος όχι στα \textlatin{pixel} της εικόνας αλλά στο διάνυσμα των χαρακτηριστικών το οποίο προέκυψε απο την εφαρμογή του αλγορίθμου εξαγωγής χαρακτηριστικών \textlatin{HoG}. Και σε αυτή την περίπτωση φάνηκε ξεκάθαρα απο τα πειράματα ότι μπορούμε να μειώσουμε κατά έναν μεγάλο βαθμό τον όγκο της πληροφορίας την οποία θα πρέπει να επεξεργαστούμε ώστε να ταξινομήσουμε τα δεδομένα στις κατάλληλες κλάσεις. Επίσης σε συγκεκριμένες περιπτώσεις, ακόμα και μετά απο δραματική μείωση της διάστασης του διανύσματος χαρακτηριστικών \textlatin{HoG} το αποτέλεσμα είναι εξίσου καλό ή και καλύτερο απο αυτό του χώρου των αρχικών διαστάσεων. Αυτό το αποτέλεσμα φανερώνει την δυνατότητα του αλγορίθμου να εντοπίζει και να απομακρύνει τον θόρυβο που περιέχεται στην πληροφορία των αρχικών χαρακτηριστικών βελτιώνοντας έτσι και τον χρόνο των υπολογισμών αλλά και την απόδοση του τελικού αποτελέσματος. Απο το τελευταίο πείραμα στο οποίο επίσης ο αλγόριθμος εφαρμόζεται σε έναν πολύ μεγάλο αριθμό χαρακτηριστικών-παραμέτρων (10.000) φαίνεται ότι και σε αυτή την περίπτωση εντοπίζονται τα στοιχεία τα οποία δεν μπορούν να συνεισφέρουν θετικά στην εξαγωγή ορθού συμπεράσματος και αποβάλλοντάς τα επιτυγχάνεται πολύ μεγάλη αύξηση στο ποσοστό ορθής ταξινόμησης των δεδομένων.
\par
Απο την εκτέλεση των παραπάνω πειραμάτων μπορούν επίσης να εξαγχθούν και συγκεκριμένα στοιχεία ως προς τον τρόπο λειτουργίας του αλγορίθμου \textlatin{LLE}. Πιο συγκεκριμένα απο το πρώτο πείραμα μπορούμε να συμπεράνουμε ότι αντίθετα με αλγορίθμους μηχανικής μάθησης όπως τα Νευρωνικά δίκτυα, για την εκπαίδευση του αλγορίθμου δεν απαιτείται τεράστιος αριθμός δεδομένων αλλά αρκεί ένας σωστά δειγματοληπτημένος χώρος ο οποίος να διατηρεί το λείο της πολλαπλότητας. Επίσης ο χώρος των δεδομένων θα πρέπει να είναι ομοιόμορφα δειγματοληπτημένος ώστε να μην υπάρχουν <<μεγάλες αποστάσεις>> μεταξύ δεδομένων της ίδιας κλάσης διότι αυτό μπορεί να αποτελέσει αρνητικό παράγοντα στην διατήρηση των γεωμετρικών χαρακτηριστικών της γειτονιάς για κάποιο σημείο.
\par
Απο τα δύο επόμενα πειράματα φάνηκε ότι ο αλγόριθμος \textlatin{LLE} είναι σε θέση να επιτύχει μείωση των παραμέτρων σε διανύσματα χαρακτηριστικών γεγονός το οποίο αποτελεί σημαντική μείωση του κόστους των υπολογισμών. Και αυτό διότι η εξαγωγή χαρακτηριστικών σε εικόνες είναι απο μόνος του ένας τρόπος μείωσης κατά ένα μεγάλο ποσοστό του κόστους των υπολογισμών. Με την μείωση λοιπόν των παραμέτρων του διανύσματος των χαρακτηριστικών για μια εικόνα η συμπίεση της πληροφορίας πλέον είναι τεράστια και το αποτέλεσμα της ταξινόμησης μπορεί πλέον να εξαχθεί λαμβάνοντας υπόψιν έναν πολύ μικρό αριθμό παραμέτρων. 
\par
Όπως είχε αναλυθεί και στην εισαγωγή της εργασίας η διαδικασία αυτή, δηλαδή η δραματική μείωση των παραμέτρων που πρέπει να εκτιμηθούν για την εξαγωγή κάποιου αποτελέσματος δόθηκε σαν ερέθισμα στον χώρο της Τεχνητής νοημοσύνης απο τον τρόπο με τον οποίο λειτουργεί ο ανθρώπινος εγκέφαλος. Αδιαμφισβήτητα λοιπόν, και λαμβάνοντας υπόψιν τα θετικά αποτελέσματα των πειραμάτων της εν λόγω εργασίας σε εφαρμογές αναγνώρισης προτύπων, η μείωση των διαστάσεων θα πρέπει να αποτελεί βασικό κομμάτι προεπεξεργασίας της πληροφορίας σε όλες σχεδόν τις εφαρμογές Μηχανικής μάθησης. Όπως έγινε φανερό με τον τρόπο αυτό μπορεί να μειωθεί δραματικά το κόστος των υπολογισμών με αποτέλεσμα να είναι εφικτή η εξαγωγή συμπεράσματος για εφαρμογές αναγνώρισης προτύπων σε πραγματικό χρόνο. Επίσης μπορεί να χρησιμοποιηθεί η μείωση των διαστάσεων σε Ιατρικές εφαρμογές επιτυγχάνοντας όπως φάνηκε απο το συγκεκριμένο πείραμα πολύ μεγάλη βελτίωση στην πρόβλεψη εμφάνισης συγκεκριμένης μορφής ασθενειών.
\par
Κλείνοντας λοιπόν την εν λόγω διατριβή είναι φανερή η αποτελεσματικότητα του συγκεκριμένου αλγορίθμου \textbf{(\textlatin{LLE})} μείωσης των διαστάσεων σε εφαρμογές αναγνώρισης προτύπων. Επίσης λαμβάνοντας υπόψιν ευρήματα ερευνών απο τον χώρο της Ιατρικής για τον τρόπο λειτουργίας του ανθρώπινου εγκεφάλου γίνεται άμεσα φανερό ότι θα πρέπει να μπορέσουμε, αφού κατανοήσουμε πλήρως τον τρόπο με τον οποίο λειτουργεί, στην συνέχεια να εφαρμόσουμε αντίστοιχες τεχνικές σε εφαρμογές Τεχνητής Νοημοσύνης. Οι τεχνικές αυτές σε συνδυασμό με την ραγδαία αύξηση της τεχνολογίας και των διαθέσιμων πόρων που μας προσφέρουν οι σημερινοί, πόσω μάλλον οι μελλοντικοί, ηλεκτρονικοί υπολογιστές μπορούν να συμβάλλουν σημαντικά στον χώρο της Ιατρικής προσφέροντας εξαιρετική βελτίωση στην αποτελεσματικότητα αντιμετώπισης σοβαρών ασθενειών. Επίσης μπορούν να χρησιμοποιηθούν σε ένα μεγάλο πλήθος τόσο καθημερινών όσο και βιομηχανικών εφαρμογών οι οποίες θα αλλάξουν κατά πολύ τον ρόλο της Ρομποτικής αλλά και αντίστοιχα της Τεχνητής Νοημοσύνης στην καθημερινότητα μας.
