
\chapter{Εισαγωγή}

\section{Αναγνώριση προτύπων και μηχανική μάθηση}
\par
Αναγνώριση προτύπων καλείται η επιστημονική περιοχή που έχει στόχο την ταξινόμηση αντικειμένων σε κατηγορίες ή κλάσεις. Ανάλογα με την κάθε εφαρμογή τα δεδομένα μπορεί να είναι είτε εικόνες, είτε σήματα είτε οποιοδήποτε άλλο σετ δεδομένων χρειάζεται για κάποιο λόγο να ταξινομηθεί. Στις μέρες μας η ανάγκη διαχείρισης αλλά και ανάκτησης πληροφοριών μέσω ηλεκτρονικών υπολογιστών αποκτά τεράστια σπουδαιότητα. Αυτό διότι ο όγκος των πληροφοριών αυξάνεται ραγδαία με ρυθμό αδύνατο να διαχειριστεί ο άνθρωπος. Επίσης η ανάπτυξη της τεχνολογίας μας παρέχει πολύ ισχυρά υπολογιστικά συστήματα με τη χρήση των οποίων μπορούμε να δημιουργήσουμε πολύπλοκα μοντέλα εξόρυξης γνώσης. 
\par
Επιστημονικοί κλάδοι στους οποίους έχει τεράστια σημασία η αναγνώριση προτύπων είναι αυτός της Ιατρικής, της Βιολογίας, ο χώρος των αγορών και των επιχειρήσεων και τέλος η διαχείριση και η εξόρυξη γνώσης απο τον τεράστιο όγκο της πληροφορίας που είναι διαθέσιμος στο διαδίκτυο. Φυσικά η αναγνώριση προτύπων είναι ένα πολύ σημαντικό μέρος του κλάδου της Μηχανικής μάθησης σε ρομποτικά/υπολογιστικά συστήματα.
\par
Η υπολογιστική όραση για παράδειγμα είναι αντικείμενο ιδιαίτερα χρήσιμο τόσο στον χώρο της Ρομποτικής όσο σε αυτόν της Ιατρικής αλλά και προφανώς της Βιομηχανίας. Τέτοιου είδους εφαρμογές έχουν εισέλθει πολύ δυναμικά στην καθημερινότητά μας τα τελευταία χρόνια. Συγκεκριμένα στον χώρο της βιομηχανίας υπάρχουν συστήματα τα οποία επιβλέπουν μέσω μια κάμερας την γραμμή παραγωγής καθώς και ρομπότ τα οποία μεταφέρουν και συναρμολογούν αντικείμενα. Eπίσης υπάρχουν εφαρμογές οι οποίες αναγνωρίζουν για παράδειγμα πρόσωπα τραβώντας μια εικόνα με το κινητό μας τηλέφωνο. Τέλος στον χώρο της αυτοκινητοβιομηχανίας δεν είναι λίγες αντίστοιχες εφαρμογές οι οποίες έχουν συμβάλει δυναμικά στην αυτόνομη οδήγηση αλλά και στην προειδοποίηση για εμπόδια κλπ.
\par
Ιδιαίτερη έμφαση αξίζει να δωθεί στην εξόρυξη γνώσης σε κλάδους όπως στη Βιολογία αλλά και στην Ιατρική. Για παράδειγμα η πρόβλεψη εμφάνισης ασθενειών όπως ο καρκίνος μέσω αναγνώρισης συγκεκριμένων μοτίβων σε εικόνες απο μαγνητικό τομογράφο, η μελέτη της αλύσίδας του γεννετικού υλικού αλλά και εγχειρίσεις υψηλής ακρίβειας με τη χρήση ρομποτικού βραχίονα.

\section{Ερεθίσματα απο τον τρόπο λειτουργίας του ανθρώπινου εγκεφάλου}
\par
Απο μελέτες που έχουν γίνει για την λειτουργία του ανθρώπινου εγκεφάλου γνωρίζουμε ότι για οποιοδήποτε σύνολο μετρήσεων προέρχεται για παράδειγμα είτε απο την όραση μας είτε απο την ακοή μας ο εγκέφαλός μας μετασχηματίζει το σύνολο των δεδομένων αυτών σε ένα νέο σύνολο χαρακτηριστικών. Με τον τρόπο αυτό, επιλέγοντας προφανώς κάθε φορά τα κατάλληλα χαρακτηριστικά, επιτυγχάνεται τεράστια συμπίεση του όγκου της πληροφορίας σε σύγκριση με τα αρχικά δεδομένα εισόδου. Αυτο σημαίνει λοιπόν ότι το μεγαλύτερο μέρος της πληροφορίας για παράδειγμα μια σκηνής που βλέπουμε και στην οποία θέλουμε να αναγνωρίσουμε τα αντικείμενα που περιέχονται, συμπιέζεται σε έναν πολύ μικρό αριθμό χαρακτηριστικών. Η παραπάνω διαδικασία χαρακτηρίζεται ως τεχνική μείωσης διάστασης γνωστή στην βιβλιογραφία με τον όρο \textlatin{Dimensionality Reduction}.
\par
Ας πάρουμε για παράγειγμα τον κλάδο της υπολογιστικής όρασης ο οποίος αποτελεί και αντικείμενο μελέτης της εν λόγω εργασίας και ας αναρωτηθούμε το εξής: Πόσο δύσκολο είναι για κάποιον απο εμάς να ανγνωρίσει κάποιο νούμερο αποτυπωμένο σε μια εικόνα; Η προφανής απάντηση είναι καθόλου. Και αυτή είναι μια πολύ σωστή απάντηση, διότι για τον ανθρώπινο εγκέφαλο το να καταλάβει οτι το ψηφίο το οποίο βρίσκεται στην εικόνα είναι για παράδειγμα το 1 και όχι το 9 είναι ένα πολύ απλό πρόβλημα. 
\par
Πιο συγκεκριμένα βλέποντας μια οποιαδήποτε σκηνή ο ανθρώπινος εγκέφαλος προσπαθεί να εντοπίσει σημεία ενδιαφέροντος τα οποία αποτελούν χαρακτηριστικά σημεία της. Τέτοια μπορεί να είναι πολύ έντονες αλλαγές στην φωτεινότητα όπως για παράδειγμα γωνίες, κενά ή τρύπες. Στην συνέχεια εντοπίζει πιο σύνθετες γεωμετρίες όπως ευθείες ή καμπύλες γραμμές και τέλος προσδιορίζει πιο ολοκληρωμένες δομές τρισδιάστατων αντικειμένων. Το ίδιο ακριβώς γίνεται και στην παραπάνω περίπτωση με το ψηφίο. Εντοπίζουμε αρχικά οτι το μοτίβο του ψηφίου 1 είναι πολύ κοντά σε αυτά των ψηφίων εφτά και τέσσερα αλλά σε καμιά περίπτωση δεν θα λέγαμε οτι έχει τρομερές ομοιότητες με αυτό του δύο ή του οχτώ για παράδειγμα. 
\par
Το παραπάνω παράδειγμα είναι ένα πολύ απλό δείγμα του τρόπου με τον οποίο ο ανθρώπινος εγκέφαλος προσπαθεί με κάθε τρόπο να ελαχιστοποιήσει τις παραμέτρους που πρέπει να εκτιμήσει. Φυσικά για ένα ρεαλιστικό περίπλοκο πρόβλημα της καθημερινότητάς μας θα δούμε οτι απαιτούνται πολύ πιο σύνθετοι υπολογισμοί και θα πρέπει να συνδιάσουμε ένα πλήθος απο παραμέτρους ώστε τελικά να καταλήξουμε στο τελικό συμπέρασμα για κάποια απόφαση. Σε κάθε περίπτωση όμως γίνεται τεράστια συμπίεση της αρχικής πληροφορίας μέσω τεχνικών μείωσης διαστάσεων ώστε να ελαχιστοποιηθούν οι παράμετροι που πρέπει να υπολογιστούν και προφανώς να επιταχυνθεί η διαδικασία εξαγωγής της τελικής μας απόφασης.
\par Το γεγονός αυτό και δεδομένου οτι το όραμα της επιστημονικής κοινότητας των Μηχανικών που ασχολούνται με την Μηχανική μάθηση και την Εξόρυξη Γνώσης είναι να δημιουργηθεί ένα μοντέλο αντίστοιχο με αυτό του ανθρώπινου εγκεφάλου δεν θα μπορούσε να τους αφήσει αδιάφορους ώστε να μελετήσουν και να αναπτύξουν αντίστοιχους αλγορίθμους.

\subsection{Μάθηση με επίβλεψη - χωρίς επίβλεψη - με ημιεπίβλεψη}
\par
Ένα πολύ εύλογο ερώτημα το οποίο προκύπτει απο την παραπάνω ανάλυση είναι, πως ο ανθρώπινος εγκέφαλος έχει μάθει και τελικώς έχει αποθηκεύσει το σύνολο αυτών των μοντέλων για το κάθε ψηφίο ή για οποιοδήποτε άλλο αντικείμενο ή μοτίβο μπορεί να αναγνωρίσει με τόσο μεγάλη ταχύτητα και ευκολία. Η απάντηση είναι προφανώς η συνεχής εκπαίδευση και η διαρκής υπενθύμιση των συγκεκριμένων προτύπων.
\par
Πιο συγκεκριμένα ο άνθρωπος απο την μέρα που αρχίζει να αλληλεπιδρά με το περιβάλλον παίρνει διάφορα ερεθίσματα τα οποία καιρό με τον καιρό μαθαίνει να τα ταξινομεί κατάλληλα και να τα χρησιμοποιεί σε περίπτωση που εμφανιστούν μπροστά του. Τα ερεθίσματα αυτά είναι είτε εικόνες, είτε ήχοι είτε ερεθίσματα τα οποία μπορεί να προέρχονται απο τις υπόλοιπες αισθήσεις του.
\par
Ο τρόπος με τον οποίο καταφέρνουμε να συγκρατούμε και να μπορούμε να διαχειριστούμε ανα πάσα στιγμή τον τεράστιο όγκο πληροφοριών που βρίσκονται καταχωρημένες στον εγκέφαλό μας είναι ένας συνδιασμός τεχνικών μάθησης και συνεχούς εκπαίδευσης. Οι τεχνικές αυτές στον χώρο της τεχνητής νοημοσύνης αναφέρονται ως τεχνικές μάθησης με επίβλεψη, χωρίς επίβλεψη και με ημιεπίβλεψη. Θα μπορούσε κάποιος αρχικά να υποστηρίξει ότι ο ανθρώπινος εκγέφαλος χρησιμοποιεί κατεξοχήν τεχνικές μάθησης χωρίς επίβλεψη διότι μπορεί να μαθαίνει μόνος του νέα πράγματα. Είναι όμως πραγματικά αυτό το οποίο συμβαίνει; Η απάντηση είναι όχι, και αυτό διότι απο την πολύ νεαρή του υλικία ο καθένας μας έχει γύρω του ανθρώπους οι οποίοι προσπαθούν συνεχώς να μας μεταφέρουν γνώση και να μας μάθουν τι βρίσκεται γύρω μας και πως να αλληλεπιδρούμε μαζί του. Παρ'όλα αυτά μετά απο κάποιο σημείο ο ανθρώπινος εγκέφαλος αποκτά δυνατότητες με τις οποίες μπορεί να αξιολογεί και να μαθαίνει μόνος του πολύ σύνθετα προβλήματα. Αυτό το επιτυγχάνει αναλύοντάς τα σε απλούστερα τα οποία γνωρίζει ήδη πως να τα διαχειριστεί. Επίσης είναι στην φύση του ανθρώπου να εξερευνεί συνεχώς άγνωστα μονοπάτια και να αναζητεί απαντήσεις σε άγνωστα προβλήματα επιτυγχάνοντας αξιοθαύμαστα αποτελέσματα.
\par
Απο τα παραπάνω καταλήγουμε στο συμπέρασμα ότι ο άνθρωπος χρησιμοποιεί τεχνικές ημιεπίβλεψης για την εκπαίδευση του εγκεφάλου του γεγονός το οποίο του δίνει την δυνατότητα να μπορεί να διαχειριστεί αλλά και να μάθει πολύ σύνθετα μοντέλα. Μέσα απο αυτή την διαδικασία είναι σε θέση με το πέρασμα του χρόνου να δημιουργήσει ένα τεράστιο και πανίσχυρο δίκτυο πληροφοριών, ταξινομημένο με τρόπο τον οποίο δεν μπορούμε ακόμα να εξηγήσουμε και να κατανοήσουμε. Με αυτό το μοντέλο είναι σε θέση ταχύτατα να αποφασίζει που βρίσκεται ο ευρύτερος χώρος της πληροφορίας που θέλει να αντλήσει και στην συνέχεια να αποφασίζει με τεράστια ακρίβεια και ταχύτητα την τελική του απόφαση.   
\par
Το μοντέλο αυτό είναι αν μη τι άλλο αξιοθαύμαστο και μέχρι στιγμής ανεξήγητο. Παρ' όλα είναι πολύ δύσκολο να εφαρμοστεί στον τομέα της τεχνητής νοημοσύνης και αυτό διότι ακόμα δεν είμαστε σε θέση να δώσουμε εξηγήσεις για τον ακριβή τρόπο λειτουργίας του. Το συνηθέστερο και πιο αποτελεσματικό μέχρι στιγμής μοντέλο το οποίο χρησιμοποιείται στην εξόρυξη γνώσης μέσω ηλεκτρονικών υπολογιστών είναι αυτό της μάθησης με επίβλεψη. Σύμφωνα με το μοντέλο αυτό θα πρέπει αν συλλέξουμε ένα μεγάλο συνήθως όγκο δεδομένων τον οποίο να τροφοδοτήσουμε στην συνέχεια ως είσοδο στο σύστημά μας και με την κατάλληλη μεθοδολογία να το καθοδηγήσουμε ώστε τελικά να μάθει συγκεκριμένα μοντέλα τα οποία να μπορεί να χρησιμοποιήσει στην συνέχεια με σκοπό την εξαγωγή κάποιου συμπεράσματος.

\section{Μείωση της διάστασης των δεδομένων}
\par
Στην παραπάνω διαδικασία δεδομένου ότι στις περισσότερες περιπτώσεις έχουμε να αντιμετωπίσουμε πολύ σύνθετα υπολογιστικά προβλήματα ο αριθμός των παραμέτρων που πρέπει να υπολογιστούν είναι σε συγκεκριμένες εφαρμογές απαγορευτικά μεγάλος. Σε κάποιες εφαρμογές το πρόβλημα είναι θέμα χρόνου όπου πρέπει να γίνει μείωση των παραμέτρων ώστε να ελαχιστοποιηθεί ο χρόνος εξαγωγής του συμπεράσματος. Σε άλλες είναι θέμα χώρου διότι ένας μεγάλος αριθμός πολυδιάστατων δεδομένων μπορεί να αποτελεί πρόβλημα σε συγκεκριμένες εφαρμογές. Τέλος υπάρχουν περιπτώσεις στις οποίες χρειαζόμαστε την μείωση των διαστάσεων ώστε να διώξουμε εντελώς παραμέτρους οι οποίες επιδρούν σαν θόρυβος και επηρεάζουν αρνητικά την εξαγωγή ορθού συμπεράσματος ταξινόμησης. Προφανώς σε πολλές πρακτικές εφαρμογές επικρατεί ένας συνδυασμός των παραπάνω αναγκών.
\par
Αντικείμενο λοιπόν της εν λόγω διπλωματικής εργασίας είναι η διερεύνηση και η χρήση του αλγορίθμου \textlatin{Locally Linear Embeddings} για την μείωση των διαστάσεων σε πρακτικά προβλήματα όπως η αναγνώριση ψηφίων αλλά και η ταξινόμηση ασθενών με βάση το αν πρόκειται να εμφανίσουν κάποιας μορφής καρκίνο ή όχι. Τα αποτελέσματα των πειραμάτων είναι ιδιαίτερα ενθαρυντικά και δείχνουν σε όλες τις περιπτώσεις ότι η μείωση των διαστάσεων επιδρά δραματικά στην μείωση του κόστους των υπολογισμών αλλά και στην αύξηση της σωστής πρόβλεψης λόγω απομάκρυνσης του θορύβου. Επίσης παρουσιάζονται δύο πρακτικές και ρεαλιστικές μέθοδοι εφαρμογής του αλγορίθμου σε πραγματικά προβλήματα απο τις οποίες η πρώτη παρέχει την δυνατότητα για την ταξινόμηση των αποτελεσμάτων και την εξαγωγή συμπεράσματος σε πραγματικό χρόνο και η δεύτερη έρχεται να αντιμετωπίσει το πρόβλημα της πολύ μεγάλης υπολογιστικής πολυπλοκότητας που απαιτεί η εκτέλεση του τελευταίο βήματος του αλγορίθμου.

