
%\addcontentsline{toc}{chapter}{Αλγόριθμοι μείωσης διαστάσεων}

\chapter{Αλγόριθμοι μείωσης διαστάσεων}

\section{Μαθηματικό και θεωρητικό υπόβαθρο}
\par
Έστω για παράδειγμα οτι έχουμε ένα σύνολο δειγμάτων εισόδου με αντίστοιχο διάνυσμα \textlatin{\textbf{x}} διάστασης $N\times1$,
\vspace*{0.1mm}
\begin{center}
$x^{T} = [x(0),...,x(N-1)]$
\end{center}

\section{Αλγόριθμοι για γραμμική μείωση διαστάσεων}
\subsection{\textlatin{PCA}}
\subsection{\textlatin{MDS}}

\section{Αλγόιθμοι για μη γραμμική μείωση διαστάσεων}
\subsection{\textlatin{ISOMAP}}
\subsection{\textlatin{Laplassian Eigenmaps}}
\subsection{\textlatin{LLE}}
